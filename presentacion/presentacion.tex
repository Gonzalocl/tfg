%%%%%%%%%%%%%%%%%%%%%%%%%%%%%%%%%%%%%%%%%
% Beamer Presentation
% LaTeX Template
% Version 1.0 (10/11/12)
%
% This template has been downloaded from:
% http://www.LaTeXTemplates.com
%
% License:
% CC BY-NC-SA 3.0 (http://creativecommons.org/licenses/by-nc-sa/3.0/)
%
%%%%%%%%%%%%%%%%%%%%%%%%%%%%%%%%%%%%%%%%%

%----------------------------------------------------------------------------------------
%	PACKAGES AND THEMES
%----------------------------------------------------------------------------------------

\documentclass{beamer}

\mode<presentation> {

% The Beamer class comes with a number of default slide themes
% which change the colors and layouts of slides. Below this is a list
% of all the themes, uncomment each in turn to see what they look like.

%\usetheme{default}
%\usetheme{AnnArbor}
%\usetheme{Antibes}
%\usetheme{Bergen}
%\usetheme{Berkeley}
%\usetheme{Berlin}
%\usetheme{Boadilla}
%\usetheme{CambridgeUS}
%\usetheme{Copenhagen}
%\usetheme{Darmstadt}
%\usetheme{Dresden}
%\usetheme{Frankfurt}
%\usetheme{Goettingen}
%\usetheme{Hannover}
%\usetheme{Ilmenau}
%\usetheme{JuanLesPins}
%\usetheme{Luebeck}
\usetheme{Madrid}
%\usetheme{Malmoe}
%\usetheme{Marburg}
%\usetheme{Montpellier}
%\usetheme{PaloAlto}
%\usetheme{Pittsburgh}
%\usetheme{Rochester}
%\usetheme{Singapore}
%\usetheme{Szeged}
%\usetheme{Warsaw}

% As well as themes, the Beamer class has a number of color themes
% for any slide theme. Uncomment each of these in turn to see how it
% changes the colors of your current slide theme.

%\usecolortheme{albatross}
%\usecolortheme{beaver}
%\usecolortheme{beetle}
%\usecolortheme{crane}
%\usecolortheme{dolphin}
%\usecolortheme{dove}
%\usecolortheme{fly}
%\usecolortheme{lily}
%\usecolortheme{orchid}
%\usecolortheme{rose}
%\usecolortheme{seagull}
%\usecolortheme{seahorse}
%\usecolortheme{whale}
%\usecolortheme{wolverine}

%\setbeamertemplate{footline} % To remove the footer line in all slides uncomment this line
%\setbeamertemplate{footline}[page number] % To replace the footer line in all slides with a simple slide count uncomment this line

%\setbeamertemplate{navigation symbols}{} % To remove the navigation symbols from the bottom of all slides uncomment this line
}

\usepackage{graphicx} % Allows including images
\usepackage{booktabs} % Allows the use of \toprule, \midrule and \bottomrule in tables

%----------------------------------------------------------------------------------------
%	TITLE PAGE
%----------------------------------------------------------------------------------------

\title[Clasificación de imágenes médicas]{Análisis y clasificación de imágenes médicas mediante redes neuronales} % The short title appears at the bottom of every slide, the full title is only on the title page

\author{Gonzalo Caparrós Laiz} % Your name
\institute[UM] % Your institution as it will appear on the bottom of every slide, may be shorthand to save space
{
Universidad de Murcia \\ % Your institution for the title page
\medskip
\textit{gonzalo.caparrosl@um.es} % Your email address
}
\date{Junio, 2020} % Date, can be changed to a custom date

\begin{document}

\begin{frame}
\titlepage % Print the title page as the first slide
\end{frame}

%\begin{frame}
%\frametitle{Índice} % Table of contents slide, comment this block out to remove it
%\tableofcontents % Throughout your presentation, if you choose to use \section{} and \subsection{} commands, these will automatically be printed on this slide as an overview of your presentation
%\end{frame}

%----------------------------------------------------------------------------------------
%	PRESENTATION SLIDES
%----------------------------------------------------------------------------------------

\begin{frame}
\frametitle{Inteligencia Artificial}

\begin{block}{IA}
Programas que intentan resolver problemas, simulando el comportamiento humano.
\end{block}

\begin{itemize}
\item Machine learning: es un tipo de inteligencia artificial en la que al algoritmo no se le especifica cómo resolver la tarea que se pretende resolver, sino que el algoritmo aprende.
\begin{itemize}
\item sup
\item nos
\item ref
\end{itemize}
\item NN
\end{itemize}

\end{frame}



\begin{frame}
\frametitle{Tipos de capas}

\begin{itemize}
\item Capa totalmente conectada
\item Conv
\item Conv dilatadas
\item Concatenación, pool, dropout softmax
\end{itemize}

\end{frame}



\begin{frame}
\frametitle{Técnicas}

\begin{itemize}
\item TL
\item RF
\item Cascada
\end{itemize}

\end{frame}



\begin{frame}
\frametitle{Distorsión}

\begin{block}{Distorsión}
Distorsión
\end{block}

\begin{itemize}
\item Características
\begin{itemize}
\item
\item
\item
\end{itemize}
\item Ventajas
\end{itemize}

\end{frame}



\begin{frame}
\frametitle{Bottleneck}

\begin{block}{Bottleneck}
Bottleneck
\end{block}

\end{frame}



\begin{frame}
\frametitle{Entrenamiento completo}

\end{frame}



\begin{frame}
\frametitle{Añadir una capa}

\end{frame}



\begin{frame}
\frametitle{Añadir 3 capas}

\end{frame}



\begin{frame}
\frametitle{Quitar capas}

\end{frame}



\begin{frame}
\frametitle{Añadir una capa}
Añadir una capa pero bien

\end{frame}



\begin{frame}
\frametitle{Usar un slice}

\end{frame}



\begin{frame}
\frametitle{Cascada}
Subsets

\end{frame}



\begin{frame}
\frametitle{Resultados cascada}

\end{frame}



\begin{frame}
\frametitle{Inferencia por paciente}

\end{frame}



\begin{frame}
\frametitle{Distorsión}

\end{frame}



\begin{frame}
\frametitle{Resultado final}

\end{frame}



\begin{frame}
\frametitle{Vías futuras}

\end{frame}



\begin{frame}
\Huge{\centerline{Fin}}
\end{frame}

\end{document} 