% Configuración del documento y paquetes
\documentclass[12pt,a4paper]{article}
\usepackage[english,spanish]{babel}
\usepackage[utf8]{inputenc}
\usepackage[T1]{fontenc}
\usepackage{amsmath}
\usepackage{svg}
\usepackage{graphicx}
\usepackage[colorinlistoftodos]{todonotes}
\usepackage[top=1.5cm, bottom=2.0cm, left=1.50cm, right=1.50cm]{geometry}
\usepackage[nottoc]{tocbibind}
\usepackage{amssymb}
\usepackage{hyperref}
\usepackage{textcomp}
\usepackage{multicol} 
\usepackage{longtable} 
\usepackage[printonlyused,withpage]{acronym}
\usepackage[justification=centering]{caption}
\usepackage{afterpage}
\usepackage{subfig}
\usepackage[export]{adjustbox}
\usepackage{float}

\usepackage{appendix}
\renewcommand{\appendixname}{Anexos}
\renewcommand{\appendixtocname}{Anexos}
\renewcommand{\appendixpagename}{Anexos}

\usepackage{listings}
\usepackage{color}
\definecolor{codegreen}{rgb}{0,0.6,0}
\definecolor{codegray}{rgb}{0.5,0.5,0.5}
\definecolor{codepurple}{rgb}{0.58,0,0.82}
\definecolor{backcolour}{rgb}{0.95,0.95,0.92}
\definecolor{gray75}{gray}{.75}
\lstdefinestyle{mystyle}{
    backgroundcolor=\color{backcolour},   
    commentstyle=\color{codegreen},
    keywordstyle=\color{magenta},
    numberstyle=\tiny\color{codegray},
    stringstyle=\color{codepurple},
    basicstyle=\footnotesize,
    breakatwhitespace=false,         
    breaklines=true,                 
    captionpos=b,                    
    keepspaces=true,                 
    numbers=left,                    
    numbersep=5pt,                  
    showspaces=false,                
    showstringspaces=false,
    showtabs=false,                  
    tabsize=2,
    literate=
  {á}{{\'a}}1 {é}{{\'e}}1 {í}{{\'i}}1 {ó}{{\'o}}1 {ú}{{\'u}}1
  {Á}{{\'A}}1 {É}{{\'E}}1 {Í}{{\'I}}1 {Ó}{{\'O}}1 {Ú}{{\'U}}1
  {à}{{\`a}}1 {è}{{\`e}}1 {ì}{{\`i}}1 {ò}{{\`o}}1 {ù}{{\`u}}1
  {À}{{\`A}}1 {È}{{\'E}}1 {Ì}{{\`I}}1 {Ò}{{\`O}}1 {Ù}{{\`U}}1
  {ä}{{\"a}}1 {ë}{{\"e}}1 {ï}{{\"i}}1 {ö}{{\"o}}1 {ü}{{\"u}}1
  {Ä}{{\"A}}1 {Ë}{{\"E}}1 {Ï}{{\"I}}1 {Ö}{{\"O}}1 {Ü}{{\"U}}1
  {â}{{\^a}}1 {ê}{{\^e}}1 {î}{{\^i}}1 {ô}{{\^o}}1 {û}{{\^u}}1
  {Â}{{\^A}}1 {Ê}{{\^E}}1 {Î}{{\^I}}1 {Ô}{{\^O}}1 {Û}{{\^U}}1
  {œ}{{\oe}}1 {Œ}{{\OE}}1 {æ}{{\ae}}1 {Æ}{{\AE}}1 {ß}{{\ss}}1
  {ű}{{\H{u}}}1 {Ű}{{\H{U}}}1 {ő}{{\H{o}}}1 {Ő}{{\H{O}}}1
  {ç}{{\c c}}1 {Ç}{{\c C}}1 {ø}{{\o}}1 {å}{{\r a}}1 {Å}{{\r A}}1
  {€}{{\euro}}1 {£}{{\pounds}}1 {«}{{\guillemotleft}}1
  {»}{{\guillemotright}}1 {ñ}{{\~n}}1 {Ñ}{{\~N}}1 {¿}{{?`}}1
}
\renewcommand{\lstlistingname}{Listado}
\lstset{style=mystyle}

\definecolor{maroon}{rgb}{0.5,0,0}
\definecolor{darkgreen}{rgb}{0,0.5,0}
\lstdefinelanguage{XML}
{
  %basicstyle=\ttfamily,
  morestring=[s]{"}{"},
  morecomment=[s]{?}{?},
  morecomment=[s]{!--}{--},
  commentstyle=\color{darkgreen},
  moredelim=[s][\color{black}]{>}{<},
  moredelim=[s][\color{red}]{\ }{=},
  stringstyle=\color{blue},
  identifierstyle=\color{maroon}
}

% Configuración de los márgenes del documento
\usepackage{vmargin}
\setpapersize{A4}
\setmargins{2.5cm} % margen izquierdo
{1.5cm}            % margen superior
{16.6cm}           % anchura del texto
{23.42cm}          % altura del texto
{16pt}             % altura de los encabezados
{1cm}              % espacio entre el texto y los encabezados
{0pt}              % altura del pie de página
{2cm}              % espacio entre el texto y el pie de página

% Configuración de cabecera y pie de página
\usepackage{fancyhdr}
\pagestyle{fancy}
\fancyhf{}
\fancyhead[L]{\nouppercase{\leftmark}}
\fancyfoot[C]{Página \thepage}
\renewcommand{\headrulewidth}{0.5pt}
\renewcommand{\footrulewidth}{0.5pt}
\hypersetup{
    colorlinks=true,
    linkcolor=black,
    filecolor=magenta,      
    urlcolor=blue,
    linkbordercolor=white
}
% diccionario de acrónimos
\usepackage[acronym,toc,shortcuts]{glossaries}
\setlength{\glsdescwidth}{0.8\textwidth}
\makeglossaries
\newacronym{GFLOPS}{GFLOPS}{Giga Floating Point Operations Per Second}
\newacronym{TFLOPS}{TFLOPS}{Tera Floating Operations Per Second}
%\newacronym{}{}{}

% Algoritmos en pseudcódigo 
\usepackage[ruled,linesnumbered,spanish,onelanguage]{algorithm2e}
\SetAlFnt{\footnotesize}
% nuevos comandos (macros)
\newcommand{\seccion}[1]{
\textcolor{blue}{
\hrule height 1.5pt
\section{#1}
\hrule height 1.5pt
\vspace{0.5cm}}
}
\newcommand{\subseccion}[1]{
\textcolor{blue}{\subsection{#1}
\hrule
\vspace{0.5cm}}
}
% Configuración de la portada del documento
\newcommand{\doctitle}{Análisis y clasificación de imágenes médicas mediante redes neuronales}
\newcommand{\docauthor}{Gonzalo Caparrós Laiz}
\newcommand{\docdate}{Junio, 2020}
\newcommand{\docdirector}{Dr. D. Gregorio Bernabé García y Dr. D. José Manuel García Carrasco}

\title{\doctitle}
\author{\docauthor}
\date{\docdate}
% Documento 
\begin{document}
\renewcommand*\listtablename{Índice de tablas}
\renewcommand\spanishtablename{Tabla}
\glsunsetall

\newcommand*{\blankpage}{%
\vspace*{\fill}
{\centering Esta página ha sido intencionadamente dejada en blanco\par}
\vspace{\fill}}
% *********************** Portada del Documento
\begin{titlepage}
\begin{center}
\begin{figure}[ht]
\centering
\includegraphics[width=0.5\textwidth]{img/escudo}
\end{figure}
\vspace{1cm}
\begin{Large}
\textbf{Universidad de Murcia\\
Facultad de Informática\\}
\vspace{0.5cm}
GRADO EN INGENIERÍA INFORMÁTICA\\
\vspace{1.0cm}
\textbf{\doctitle}
\end{Large}
\vspace{1.0cm}
\hrule
\vspace{0.2cm}
\begin{large}
\textbf{Trabajo Fin de Grado realizado por:}\\\docauthor\\
\vspace{0.2cm}
\textbf{Bajo la dirección de:}\\
\docdirector\\
\vspace{0.2cm}
\docdate\\
\end{large}
\vspace{0.5cm}
\hrule
\vspace{1cm}
\end{center}
\end{titlepage}
%\pagenumbering{roman}
% *********************** Fin Portada del Documento
\renewcommand{\headrulewidth}{0.0pt}
\renewcommand{\footrulewidth}{0.0pt}
\fancyhead[L]{}
\fancyfoot[C]{}
\newpage
\blankpage
%**************************************************
\newpage
\fancyhead[L]{\nouppercase{\leftmark}}
\fancyfoot[C]{Página \thepage}
\renewcommand{\headrulewidth}{0.5pt}
\renewcommand{\footrulewidth}{0.5pt}
\tableofcontents

\newpage
\listoffigures

\newpage
\listoftables

\newpage
%\pagenumbering{arabic}
\section*{Resumen}
\fancyhead[L]{Resumen}
\addcontentsline{toc}{section}{Resumen}

\newpage
\section*{Extended Abstract}
\fancyhead[L]{Extended Abstract}
\addcontentsline{toc}{section}{Extended Abstract}

\newpage
\section{Introducción}
\fancyhead[L]{\nouppercase{\rightmark}}

Ciertos procedimientos en medicina requieren la dedicación de muchas horas de trabajo de los profesionales del sector, por lo tanto consumen muchos recursos ya sea de tiempo, muy valioso para dar diagnósticos de forma rápida y eficaz, como económicos. Uno de estos procedimientos es el diagnóstico de ciertas enfermedades, los cuales requieren la segmentación de imágenes médicas de distintos órganos y tejidos. En concreto en este trabajo vamos a tratar de la clasificación de las Imágenes por Resonancia Magnética (\textit{MRI}) del corazón y por lo tanto su diagnóstico.
\bigskip

Estas tareas son muy repetitivas siendo esto una ventaja para los programas de inteligencia artificial ya que lo hacen siempre de la misma manera y de forma mucho más rápida que las personas. Este trabajo se basa en el paper \textit{Automatic Segmentation and Disease Classification Using Cardiac Cine MR Images} ~\cite{DBLP:journals/corr/abs-1708-01141} en el que se desarrolla una red neuronal basada en convoluciones dilatadas. En este trabajo se propone una solución al mismo problema pero con un enfoque distinto, partir de una red neuronal ya entrenada en un conjunto de datos distintos y utilizar técnicas de \textit{transfer learning} para resolver el mismo problema que en el paper ~\cite{DBLP:journals/corr/abs-1708-01141}. Se hacen diferentes pruebas y se estudia cual es la mejor solución. En otras palabras se intenta reducir el tiempo de una tarea que hecha por los procedimientos tradicionales sería muy lenta, de esta forma podemos dar un diagnóstico más rápido y a más pacientes. Para llevar a cabo el trabajo se usa el framework \textit{TensorFlow} y se utilizan los modelos de partida que se proporcionan en \textit{TensorFlow Hub}, principalmente \textit{resnet} e \textit{inceptionv3}.

\newpage
\section{Estado del arte}

\subsection{Inteligencia artificial}
Se entiende por inteligencia artificial como los programas que intentan simular el comportamiento humano para resolver un problema. Hay muchos formas de enfocar un programa de inteligencia artificial unas más fáciles de entender como la simbólica, que es un programa basado en reglas que intenta imitar el comportamiento humano, o un enfoque sin símbolos en el que se encontraría el \textit{machine learning}, en el que podemos tener millones de parámetros que entrenar y no saber qué representa cada uno.

\subsection{Machine learning}
Es un tipo de inteligencia artificial sin símbolos que consiste en algoritmos que resuelven un problema sin darle instrucciones explícitas de cómo se resuelve el problema. Estos algoritmos construyen un modelo matemático que suele tener millones de parámetros, que se entrenan, difíciles de analizar para saber qué es lo que hace que dé el resultado que da. Generalmente el \textit{machine learning} se usa para resolver problemas para los que no es factible o es muy difícil hacer un algoritmo tradicional.
\bigskip

Para ello parte de unos datos de muestra o datos de entrada de los que aprenderá a resolver el problema que tiene que resolver, mediante repetición de cálculo del error y ajuste del modelo. Estos datos se suelen dividir en tres conjuntos o sets:

\begin{itemize}
\item \textbf{\textit{Training} set}: este set se usa para que el modelo se entrene. Con este set se calculan los errores en el momento del entrenamiento y se usan para ajustar los parámetros del modelo que se está entrenando.
\item \textbf{\textit{Validation} set}: este set se usa durante el entrenamiento para evaluar el modelo y poder ver el progreso.
\item \textbf{\textit{Test} set}: este set se usa al final, cuando el modelo ya ha sido entrenado, para evaluar el modelo con unos datos que nunca hayan sido vistos de ninguna manera por el modelo.
\end{itemize}

Hay principalmente tres formas de hacer que un modelo aprenda a resolver el problema cada enfoque es más apropiado para resolver un tipo de problema.

\subsubsection{Aprendizaje supervisado}
El aprendizaje supervisado consiste en aplicar \textit{machine learning} a unos datos de entrada que están etiquetados y por lo tanto el modelo aprenderá a etiquetar esas clases. Aplicaciones comunes del aprendizaje supervisado son la clasificacion y regresion. En la clasificación la salida de los modelo es un número limitado de valores que son los que los datos de entrada están etiquetados. En la regresión la salida puede tener cualquier valor numérico en un rango.

\subsubsection{Aprendizaje no supervisado}
El aprendizaje no supervisado consiste en aplicar \textit{machine learning} a unos datos de entrada que no estén etiquetados y el modelo aprenda a encontrar patrones en esos datos y a agruparlos de forma automática.

\subsubsection{Aprendizaje por refuerzo}
El aprendizaje por refuerzo consisten en agentes que se mueven por un entorno realizando acciones obteniendo una recompensa cuando se van acercando a la solución. El objetivo básico es maximizar la recompensa. En este enfoque no es necesario unos datos de entrada etiquetados.

\subsection{Redes Neuronales}
Inspiradas en el funcionamiento de las redes neuronales biológicas, una red neuronal es una forma de organizar los modelos de inteligencia artificial para abordar ciertos problemas que son difíciles de resolver con la programación ordinaria basada en ``Reglas'' obteniendo muy buenos resultados en una amplia variedad de tareas como por ejemplo: la visión por ordenador o el reconocimiento de la voz.
\bigskip

Este tipo de inteligencia artificial se organiza en varias capas interconectadas entre sí. Las capas se suelen nombrar como capa de entrada, que es la primera capa, a la que se le pasan directamente los datos de entrada, capa de salida, que es la última capa, la que da el resultado de la predicción que hace la red neural y capas ocultas, que son las que están entre la primera y la última.
\bigskip

Las capas de una red neuronal están compuestas por neuronas artificiales.

\begin{figure}[H]
\centering
\includegraphics[width=0.3\textwidth]{example-image}
\caption{Ejemplo de red neuronal con capa de entrada, capa de salida y una capa oculta.}
\end{figure}

\subsubsection{Neuronas artificiales}
Las neuronas artificiales son la unidad básica de una red neuronal reciben una entrada y producen una salida. La entrada la reciben de la capa inmediatamente anterior a la capa a la que pertenecen y la salida es la entrada de la capa inmediatamente a continuación. El funcionamiento básico de una neurona es el siguiente: las entradas se multiplican por un peso y se suman, el valor se le aplica una función que dará la salida de la neurona, también llamada función de activación.

\begin{figure}[H]
\centering
\subfloat{{\huge $x = \frac{y}{z}$}}%
\qquad
\subfloat{{\includegraphics[valign=c, width=0.3\textwidth]{example-image} }}%
\caption{Formula de suma de pesos e imagen de una neurona.}
\end{figure}

%TODO
Donde phi es la función, wi son los pesos xi son las entradas e y es la salida.
\bigskip

%TODO
Algunas de las funciones de activación más usadas son las siguientes. En los siguientes apartados u se refiere a la entrada de la función según la fórmula:
\begin{equation}
x = \frac{y}{z}
\end{equation}

\begin{itemize}
\item \textbf{Función escalonada}: La salida de esta función es 1 si el valor de entrada es mayor que cierto umbral o 0 si está por debajo del umbral.
\begin{figure}[H]
\centering
\subfloat{{\huge $x = \frac{y}{z}$}}%
\qquad
\subfloat{{\includegraphics[valign=c, width=0.3\textwidth]{example-image} }}%
\caption{Formula y gráfica.}
\end{figure}

\item \textbf{Combinación lineal}: En esta función la salida simplemente se le suma un bias. Esta función aplica una transformación lineal a la entrada
%TODO

\item \textbf{Sigmoide}: Una función sigmoide es una función matemática que tiene la característica que su curva tiene forma de \textit{s}. Con esta función las entradas negativas tienden a cero y conforme la entrada se acerca a cero la salida va aumentando hasta que el valor de entrada es positivo, conforme se aleja de cero tiende a uno. La función más utilizada es la función logística pero otras funciones matemáticas con forma de \textit{s} también se usan como la tangente o la tangente hiperbólica.
\begin{figure}[H]
\centering
\subfloat{{\huge $x = \frac{y}{z}$}}%
\qquad
\subfloat{{\includegraphics[valign=c, width=0.3\textwidth]{example-image} }}%
\caption{Formula y gráfica.}
\end{figure}

\item \textbf{Rectificador}: Esta función se define como la parte positiva de la entrada. Es decir si la entrada es positiva esa es la salida sin embargo si la entrada es negativa la salida es cero.
\begin{figure}[H]
\centering
\subfloat{{\huge $x = \frac{y}{z}$}}%
\qquad
\subfloat{{\includegraphics[valign=c, width=0.3\textwidth]{example-image} }}%
\caption{Formula y gráfica.}
\end{figure}

\end{itemize}


\newpage
\section{Conclusión}

\newpage
\appendix
\addappheadtotoc
\appendixpage
\section{Anexo I: a1}\label{anexo1}

\newpage
\section{Anexo II: a2}\label{anexo2}

\newpage
\nocite{*}
\renewcommand\refname{Bibliografía}
\bibliographystyle{ieeetr}
\bibliography{memoria}

\clearpage

%\printglossary[type=\acronymtype]

%\newpage
\printglossary[type=\acronymtype,style=long,title={Índice de acrónimos}]

%\printglossaries

\end{document}
